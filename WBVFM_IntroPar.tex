

\documentclass{article}\usepackage[]{graphicx}\usepackage[]{color}
%% maxwidth is the original width if it is less than linewidth
%% otherwise use linewidth (to make sure the graphics do not exceed the margin)
\makeatletter
\def\maxwidth{ %
  \ifdim\Gin@nat@width>\linewidth
    \linewidth
  \else
    \Gin@nat@width
  \fi
}
\makeatother

\usepackage{Sweave}


\newenvironment{knitrout}{}{}  %just a dummy environment
\makeatletter
\newcommand\gobblepars{%
    \@ifnextchar\par%
        {\expandafter\gobblepars\@gobble}%
        {}}
\makeatother



\title{A top-down approach to projecting the impacts of development aid on carbon sequestration}
\IfFileExists{upquote.sty}{\usepackage{upquote}}{}
\begin{document}
\begin{knitrout}







  Since 1945, over \$4.9 trillion dollars of international aid has been allocated (Tierney, et al., 2011).
 To date there have been no estimates of the geographically varying impact of this aid on the carbon cycle.  \gobblepars


  We apply a geographically weighted matching method (Andam, et al., 2008) to estimate the impact of World Bank projects implemented between 2000 and 2011 on sequestered carbon, using a new, double-blind coded dataset of 47,084 World Bank project locations.\gobblepars

 Considering only carbon sequestered due to fluctuations in vegetative biomass caused by World Bank projects, we estimate that these projects have resulted in a net positive increase in sequestration totaling 1,398,229 tonnes.\gobblepars






%bibliography(style="nature")


%Summary Paragraph
%WB Global IE
%Runfola, Ben'Yishay, Tanner, Buchanan, Nagol
%Goodman, Trichler, Marty
%Acknowledge: Stewart, Kappel, Lu, Nicholson, Vijayan, Walter, Hild, Leu
\end{knitrout}
\end{document}
