%% start of file `template.tex'.
%% Copyright 2006-2013 Xavier Danaux (xdanaux@gmail.com).
%
% This work may be distributed and/or modified under the
% conditions of the LaTeX Project Public License version 1.3c,
% available at http://www.latex-project.org/lppl/.
%Version for spanish users, by dgarhdez

\documentclass[11pt,a4paper,roman]{moderncv}        % possible options include font size ('10pt', '11pt' and '12pt'), paper size ('a4paper', 'letterpaper', 'a5paper', 'legalpaper', 'executivepaper' and 'landscape') and font family ('sans' and 'roman')



% moderncv themes
\moderncvstyle{classic}                            % style options are 'casual' (default), 'classic', 'oldstyle' and 'banking'
\moderncvcolor{green}                              % color options 'blue' (default), 'orange', 'green', 'red', 'purple', 'grey' and 'black'
%\renewcommand{\familydefault}{\sfdefault}         % to set the default font; use '\sfdefault' for the default sans serif font, '\rmdefault' for the default roman one, or any tex font name
%\nopagenumbers{}                                  % uncomment to suppress automatic page numbering for CVs longer than one page

% character encoding
\usepackage[utf8]{inputenc}                       % if you are not using xelatex ou lualatex, replace by the encoding you are using
%\usepackage{CJKutf8}                              % if you need to use CJK to typeset your resume in Chinese, Japanese or Korean

% adjust the page margins
\usepackage[scale=0.75]{geometry}
%\setlength{\hintscolumnwidth}{3cm}                % if you want to change the width of the column with the dates
%\setlength{\makecvtitlenamewidth}{10cm}           % for the 'classic' style, if you want to force the width allocated to your name and avoid line breaks. be careful though, the length is normally calculated to avoid any overlap with your personal info; use this at your own typographical risks...

% personal data
\name{Dr. Daniel}{Miller Runfola}
\address{William and Mary}{Institute for the Theory and Practice of International Relations}{AidData}% optional, remove / comment the line if not wanted; the "postcode city" and and "country" arguments can be omitted or provided empty
\phone[mobile]{508-316-9109}                   % optional, remove / comment the line if not wanted
%\phone[fixed]{+2~(345)~678~901}                    % optional, remove / comment the line if not wanted
%\phone[fax]{+3~(456)~789~012}                      % optional, remove / comment the line if not wanted
\email{dsmillerrunfol@wm.edu}                               % optional, remove / comment the line if not wanted
%\homepage{www.johndoe.com}                         % optional, remove / comment the line if not wanted
%\extrainfo{additional information}                 % optional, remove / comment the line if not wanted
%\photo[64pt][0.4pt]{picture}                       % optional, remove / comment the line if not wanted; '64pt' is the height the picture must be resized to, 0.4pt is the thickness of the frame around it (put it to 0pt for no frame) and 'picture' is the name of the picture file
%\quote{Some quote}                                 % optional, remove / comment the line if not wanted

% to show numerical labels in the bibliography (default is to show no labels); only useful if you make citations in your resume
%\makeatletter
%\renewcommand*{\bibliographyitemlabel}{\@biblabel{\arabic{enumiv}}}
%\makeatother
%\renewcommand*{\bibliographyitemlabel}{[\arabic{enumiv}]}% CONSIDER REPLACING THE ABOVE BY THIS

% bibliography with mutiple entries
%\usepackage{multibib}
%\newcites{book,misc}{{Books},{Others}}
%----------------------------------------------------------------------------------
%            content
%----------------------------------------------------------------------------------
\begin{document}
%-----       letter       ---------------------------------------------------------
% recipient data
\recipient{Daniel M Kammen}{Environmental Research Letters}
\date{\today}
\opening{Dear Dr. Kammen,}
\closing{Thank you for your time, your many contributions to the field, and for your consideration,}
         % use an optional argument to use a string other than "Enclosure", or redefine \enclname
\makelettertitle

We write you to submit the attached work for consideration in Environmental Research Letters  This is an original piece, and is not being submitted to or considered for publication by any other outlet.  It is our earnest hope that this piece will aid in laying out the potential and challenges for using newly derived spatial information on climate relevant international aid - in conjunction with many existing, satellite-derived products - to better understand human-driven causal impacts on the carbon cycle.  Substantively, this work was performed in conjunction with the World Bank to analyze the effectiveness of it's environmental safeguards, and beyond the academic implications this piece should also be of considerable interest to a broader policy audience that is currently reliant on either bottom-up or "gut instinct" approaches to large-scope strategic decisions regarding aid allocation.\\
\vspace{4 mm}

\textbf{What are the new results or developments reported in your article?}\\
This work presents two key new results and developments that we believe will be of interest to the ERL readership:\\
(a) a new, freely available spatially-referenced dataset of climate-related international aid, and\\
(b) a novel approach to estimating geographically heterogeneous (causal) impact effects of international aid on vegetation and carbon sequestration through an integration of multiple, multi-decadal satellite derived measurements. \\
\vspace{4 mm}

\textbf{In what way are these new results or developments timely?}\\
The United States alone has pledged nearly \$4 billion dollars of aid to mitigate climate vulnerability over the next 4 years, and the Paris convention urged donors to target \$100 billion annually by the year 2020.  However, relatively little is known about the efficacy of this aid, especially at the global scale.  This limitation is driven in part by both a lack of information on where climate-relevant aid has been distributed, and limitations in the methods used to examine their impact on the carbon cycle at a global scale.  We hope this paper will contribute to aiding agencies in better allocating aid as they seek to meet the goals set forth in the Paris convention.\\
\vspace{4 mm}
\textbf{Why are these new results or developments significant?}\\
Relatively little attention has been given to the efficacy of climate-related aid, especially at the global scale.  The dataset presented here represents one of the first catalogues of where - and what type - of aid has been flowing to the developing world, enabling new types of causal analyses than have been historically possible.  We further implement a novel methodological strategy that allows for the estimation of impacts that may vary over geographic space - a necessity in the context of global-scope analyses.  Further, we highlight the nascent nature of research into globally-varying heterogeneous effects, and provide insight into potential next steps.
\vspace{4 mm}

\textbf{Optional. If relevant, please also include a list of two to six papers that your work extends, impacts or contradicts.}\\
\textit{This work extends the following pieces by introducing causally identified models into a spatial context:}\\
Runfola, D.M., Napier, A., 2015. “Migration, climate, and international aid: examining evidence of satellite, aid, and micro-census data.”  Accepted, Migration and Development. \\
Runfola, D.M., Romero-Lankao, P., Leiwen, J., Hunter, L., Nawrotzki, R., and Sanchez, L., 2015. “The Influence of Migration on Exposure to Extreme Weather Events: A Case Study in Mexico.” Accepted, Society and Natural Resources.\\
\textit{This work is complementary to the following piece, which examines a different methodologic approach to capturing geographic heterogeneity at global scales:}\\
Zhao, J., Runfola, D., Kemper, P., Quantifying Heterogeneous Causal Treatment Effects in World Bank Development Finance Projects. Under Review. ECML/PKDD 2016.




\vspace{0.5cm}


\makeletterclosing

\end{document}


%% end of file `template.tex'.
